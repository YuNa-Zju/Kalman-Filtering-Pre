%================================================================
\section[EKF]{扩展卡尔曼滤波 (EKF)}
%================================================================
\subsection{卡尔曼滤波的局限性}
\begin{frame}
    \frametitle{1. 线性系统的局限性}
    \begin{alertblock}{标准 KF 的致命弱点}
        标准卡尔曼滤波假设系统是\textbf{线性} 的:
        $$ \bm{x}_k = \bm{F}\bm{x}_{k-1} \quad \text{和} \quad \bm{z}_k = \bm{H}\bm{x}_k $$
    \end{alertblock}

    \textbf{然而,现实世界充满了非线性:}
    \begin{itemize}
        \item \textbf{机器人运动}:$x_{new} = x_{old} + v \cdot \cos(\theta) \cdot \Delta t$
        \item \textbf{雷达观测}:$r = \sqrt{x^2 + y^2}$
    \end{itemize}

    \vspace{0.2cm}
    \textbf{问题:} 高斯分布经过非线性变换后,\textbf{不再是高斯分布},导致标准 KF 的公式失效。
\end{frame}

\subsection{核心数学工具}

\begin{frame}
    \frametitle{2. 解决思路:线性化}
    既然非线性太难处理,我们能不能在\textbf{局部}把它看作线性的?

    \begin{block}{泰勒级数展开}
        对于非线性函数 $f(x)$,我们在估计点 $\hat{x}$ 附近做一阶展开:
        \begin{equation}
            f(x) \approx f(\hat{x}) + \underbrace{\frac{\partial f}{\partial x}\bigg|_{x=\hat{x}}}_{\text{切线斜率}} \cdot (x - \hat{x})
        \end{equation}
    \end{block}

    \begin{itemize}
        \item 我们抛弃高阶项,只保留一阶导数。
        \item 用\textbf{切线}来近似\textbf{曲线}。
    \end{itemize}
\end{frame}

\begin{frame}
    \frametitle{3. 核心工具:雅可比矩阵}
    在多维系统中,导数变成了\textbf{雅可比矩阵}。它是 EKF 的核心。

    假设状态转移函数为 $\bm{x}_k = f(\bm{x}_{k-1}, \bm{u}_k)$,则雅可比矩阵 $\bm{F}_k$ 为:

    \begin{equation}
        \bm{F}_k = \frac{\partial f}{\partial \bm{x}} =
        \begin{bmatrix}
            \frac{\partial f_1}{\partial x_1} & \frac{\partial f_1}{\partial x_2} & \cdots \\
            \frac{\partial f_2}{\partial x_1} & \frac{\partial f_2}{\partial x_2} & \cdots \\
            \vdots                            & \vdots                            & \ddots
        \end{bmatrix}
    \end{equation}

    \textbf{物理意义:} 描述了输入微小变化如何影响输出的每一个维度。
\end{frame}

\subsection{算法对比}

\begin{frame}
    \frametitle{4. EKF 算法流程的变化 (对比 KF)}
    EKF 的步骤与 KF 几乎一样,区别在于\textbf{如何传递均值}和\textbf{如何传递协方差}。

    \begin{table}
        \centering
        \small
        \begin{tabular}{l|c|c}
            \toprule
            步骤             & 标准 KF (线性)                                            & 扩展 KF (非线性)                                                                         \\
            \midrule
            \textbf{状态预测}  & $\bm{x}^- = \bm{F}\bm{x}$                             & $\bm{x}^- = \mathbf{f}(\bm{x}, \bm{u})$ \textcolor{blue}{(直接代入函数)}                  \\
            \textbf{协方差预测} & $\bm{P}^- = \bm{F}\bm{P}\bm{F}^T + \bm{Q}$            & $\bm{P}^- = \mathbf{F_k}\bm{P}\mathbf{F_k}^T + \bm{Q}$ \textcolor{red}{(用雅可比)}      \\
            \midrule
            \textbf{卡尔曼增益} & $\bm{K} = \bm{P}^-\bm{H}^T(\dots)^{-1}$               & $\bm{K} = \bm{P}^-\mathbf{H_k}^T(\mathbf{H_k}\bm{P}^-\mathbf{H_k}^T + \bm{R})^{-1}$ \\
            \textbf{状态更新}  & $\bm{x} = \bm{x}^- + \bm{K}(\bm{z} - \bm{H}\bm{x}^-)$ & $\bm{x} = \bm{x}^- + \bm{K}(\bm{z} - \mathbf{h}(\bm{x}^-))$                         \\
            \bottomrule
        \end{tabular}
    \end{table}
\end{frame}

\begin{frame}
    \frametitle{5. 经典案例:雷达追踪 (Radar Tracking)}
    \begin{columns}
        \column{0.5\textwidth}
        \textbf{状态向量 (直角坐标系):} \\
        $\bm{x} = [p_x, p_y, v_x, v_y]^T$ \\
        (飞机的位置和速度)

        \vspace{0.3cm}
        \textbf{观测向量 (极坐标系):} \\
        $\bm{z} = [\rho, \phi, \dot{\rho}]^T$ \\
        (雷达测量的距离、角度、径向速度)

        \column{0.5\textwidth}
        \textbf{非线性观测函数 $h(\bm{x})$:}
        \begin{itemize}
            \item 距离:$\rho = \sqrt{p_x^2 + p_y^2}$
            \item 角度:$\phi = \arctan(p_y / p_x)$
        \end{itemize}
    \end{columns}

    \vspace{0.5cm}
    \begin{block}{问题}
        $p_x$ 和 $p_y$ 是状态,但在观测方程里被平方和求根了。这就是典型的非线性!必须求雅可比矩阵 $\bm{H}_j$。
    \end{block}
\end{frame}

\begin{frame}
    \frametitle{6. 雷达雅可比矩阵的计算}
    我们需要对 $h(\bm{x})$ 求偏导数来得到 $\bm{H}_j$:

    \small
    \begin{equation}
        \bm{H}_j = \frac{\partial h}{\partial \bm{x}} =
        \begin{bmatrix}
            \frac{\partial \rho}{\partial p_x} & \frac{\partial \rho}{\partial p_y} & 0 & 0 \\
            \frac{\partial \phi}{\partial p_x} & \frac{\partial \phi}{\partial p_y} & 0 & 0
        \end{bmatrix}
        =
        \begin{bmatrix}
            \frac{p_x}{\sqrt{p_x^2+p_y^2}} & \frac{p_y}{\sqrt{p_x^2+p_y^2}} & 0 & 0 \\
            \frac{-p_y}{p_x^2+p_y^2}       & \frac{p_x}{p_x^2+p_y^2}        & 0 & 0
        \end{bmatrix}
    \end{equation}

    \vspace{0.3cm}
    \textbf{意义:} 这个矩阵把\textbf{位置的不确定性}(直角坐标误差),投影到了\textbf{观测的不确定性}(距离和角度误差)上。
\end{frame}

% \begin{frame}
%     \frametitle{扩展应用:不止于雷达}
%     除了经典的雷达追踪,EKF 在\textbf{信息工程}和\textbf{微电子系统}中解决了两个关键难题:

%     \vspace{0.3cm}

%     \begin{columns}[T]
%         \column{0.5\textwidth}
%         \begin{block}{微电子:电池管理系统 (BMS)}
%             \begin{itemize}
%                 \item \textbf{问题}:锂电池的剩余电量 (SOC) 无法直接测量。
%                 \item \textbf{非线性}:开路电压 (OCV) 与 SOC 呈高度非线性关系。
%                 \item \textbf{解决方案}:将 SOC 视为隐藏状态,电压视为观测,利用 EKF 估算。
%             \end{itemize}
%         \end{block}

%         \column{0.5\textwidth}
%         \begin{block}{信息工程:多传感器融合 (IMU)}
%             \begin{itemize}
%                 \item \textbf{问题}:低成本 MEMS 陀螺仪存在随机漂移 (Bias)。
%                 \item \textbf{非线性}:姿态解算中的四元数更新、三角函数运算。
%                 \item \textbf{解决方案}:误差状态卡尔曼滤波 (ES-EKF)。
%             \end{itemize}
%         \end{block}
%     \end{columns}
% \end{frame}



% %================================================================
% \subsection{信息工程应用:MEMS 姿态解算}
% %================================================================

% \begin{frame}
%     \frametitle{案例二:IMU 传感器融合的物理直觉}
%     在无人机或手机中,我们有两个互补的传感器:

%     \begin{columns}[T]
%         \column{0.5\textwidth}
%         \begin{block}{陀螺仪 (Gyro)}
%             \begin{itemize}
%                 \item \textbf{特性}:测量角速度,短时间非常准。
%                 \item \textbf{缺点}:积分会导致\textbf{漂移} (Drift)。
%                 \item \emph{类比}:闭眼走路,走越久偏得越远。
%             \end{itemize}
%         \end{block}

%         \column{0.5\textwidth}
%         \begin{block}{加速度计 (Accel)}
%             \begin{itemize}
%                 \item \textbf{特性}:测量重力方向,无累计误差。
%                 \item \textbf{缺点}:受震动干扰大,动态性能差。
%                 \item \emph{类比}:看指南针,虽然准但手会抖。
%             \end{itemize}
%         \end{block}
%     \end{columns}

%     \vspace{0.3cm}
%     \textbf{EKF 的任务:}
%     用陀螺仪做\textbf{预测}(相信它的动态),用加速度计做\textbf{观测修正}(相信它的长期稳定),把漂移拉回来。
% \end{frame}

% \begin{frame}
%     \frametitle{数学困境:为什么标准 EKF 玩不转四元数?}
%     为了避免“万向节死锁 (Gimbal Lock)”,我们使用\textbf{四元数 $\mathbf{q}$} 来表示姿态。

%     \begin{alertblock}{维度与约束的冲突}
%         \begin{itemize}
%             \item \textbf{四元数}有 4 个参数 $[w, x, y, z]$。
%             \item \textbf{自由度}只有 3 个 (Roll, Pitch, Yaw)。
%             \item \textbf{强制约束}:$|\mathbf{q}|^2 = w^2+x^2+y^2+z^2 = 1$。
%         \end{itemize}
%     \end{alertblock}

%     \textbf{如果直接用标准 EKF:}
%     \begin{itemize}
%         \item 协方差矩阵 $\bm{P}$ 是 $4 \times 4$ 的。
%         \item 由于 4 个参数不独立,$\bm{P}$ 矩阵会\textbf{奇异}(不可逆),导致滤波器崩溃。
%         \item 且每次更新后,$\mathbf{q}$ 都会不再归一化(模长变成 0.99 或 1.01),导致数学错误。
%     \end{itemize}
% \end{frame}

% \begin{frame}
%     \frametitle{核心原理:误差状态卡尔曼滤波 (ES-EKF)}
%     \textbf{解决思路:} 将“真值”拆解为两部分 —— \textbf{名义状态} + \textbf{误差状态}。

%     $$ \mathbf{x}_{true} = \mathbf{x}_{nominal} \oplus \delta \mathbf{x} $$

%     \begin{columns}
%         \column{0.55\textwidth}
%         \small
%         \begin{enumerate}
%             \item \textbf{名义状态 $\mathbf{x}_{nom}$ (非线性):}
%             \begin{itemize}
%                 \item 直接对陀螺仪数据积分。
%                 \item 不考虑噪声,累积大范围运动。
%                 \item 始终保持四元数归一化。
%             \end{itemize}

%             \item \textbf{误差状态 $\delta \mathbf{x}$ (线性):}
%             \begin{itemize}
%                 \item 定义为 3 维向量 (Roll/Pitch/Yaw 的微小偏差)。
%                 \item \textbf{符合高斯分布},完美适配 Kalman Filter。
%                 \item 初始值为 0。
%             \end{itemize}
%         \end{enumerate}

%         \column{0.45\textwidth}
%         \begin{block}{算法核心步骤}
%             \begin{enumerate}
%                 \item \textbf{预测}:积分 $\mathbf{x}_{nom}$,协方差 $\bm{P}$ 变大。
%                 \item \textbf{观测}:算出加速度计误差。
%                 \item \textbf{更新}:算出 3 维误差 $\delta \bm{\theta}$。
%                 \item \textbf{注入 (Injection)}:将误差旋转回名义状态:
%                 $$ \mathbf{q}_{new} \approx \mathbf{q}_{nom} \otimes \begin{bmatrix} 1 \\ \frac{1}{2}\delta \bm{\theta} \end{bmatrix} $$
%                 \item \textbf{重置}:将 $\delta \mathbf{x}$ 归零。
%             \end{enumerate}
%         \end{block}
%     \end{columns}

%     \footnotetext[1]{Solà, J. (2017). "Quaternion kinematics for the error-state Kalman filter". \textit{arXiv}.}
% \end{frame}
% %================================================================


%================================================================
\subsection{历史脉络与学术发展}
%================================================================

\begin{frame}
    \frametitle{EKF 的诞生与发展:从航天到芯片}
    扩展卡尔曼滤波的发展史,就是一部将理论数学转化为工程实践的历史。

    \vspace{0.2cm} % 稍微增加一点垂直间距

    \begin{columns}[T] % [T] 表示内容顶部对齐 (Top aligned)

        % --- 左边栏:文字历史 (占宽 65%) ---
        \column{0.65\textwidth}
        \begin{itemize}
            \item \textbf{1960: 线性起源} \\
                  \textit{Rudolf E. Kalman} 发表了经典论文,但仅适用于线性系统。

                  \vspace{0.3cm} % 增加列表项之间的间距,让版面更舒朗

            \item \textbf{1961: 阿波罗的困境} \\
                  NASA 的 \textit{Stanley F. Schmidt} 发现飞船轨道方程是非线性的。

                  \vspace{0.3cm}

            \item \textbf{突破点:} \\
                  Schmidt 创造性地提出了在参考轨迹附近进行\textbf{线性化}的想法,发明了 EKF。
                  \footcite{Schmidt1981History}
        \end{itemize}

        % --- 右边栏:图片与说明 (占宽 35%) ---
        \column{0.35\textwidth}
        \begin{figure}
            \centering
            % width设置为 \linewidth 表示占满当前这一栏的宽度,而不是全屏宽度
            \includegraphics[width=0.85\linewidth]{figures/Stanley_F._Schmidt.jpg}
            \caption{Stanley F. Schmidt}
        \end{figure}

    \end{columns}

    % 引用放在最底部
\end{frame}

\begin{frame}
    \frametitle{1970s-1990s: 信息工程领域的深化}

    随着计算机算力的提升,EKF 开始从航空航天下沉到一般的信号处理和控制领域。

    \begin{block}{目标追踪}
        \textbf{Yaakov Bar-Shalom} 等人完善了 EKF 在雷达追踪中的应用理论。
        \footcite{BarShalom1988Tracking}
        \begin{itemize}
            \item \emph{关键贡献}:提出了混合估计和交互式多模型。
        \end{itemize}
    \end{block}

    \begin{block}{机器人定位(SLAM)的开端}
        \textbf{Smith, Self, \& Cheeseman} (1990) 发表了开创性论文,提出用 EKF 同时估计机器人的位置和地图路标点。
        \footcite{Smith1990SLAM}
    \end{block}

    % --- 底部对应的参考文献 ---
\end{frame}

\begin{frame}
    \frametitle{2000s: 算法的收敛性证明与微电子爆发}

    \textbf{1. SLAM 的理论基石}
    \begin{itemize}
        \item 悉尼科技大学的 \textbf{Gamini Dissanayake} 等人证明了 EKF-SLAM 的解是收敛的
              \footcite{Dissanayake2001Convergence},确立了其统治地位。
    \end{itemize}

    \vspace{0.2cm}
    \textbf{2. 能源电子的革命 (微电子/电力电子)}
    \begin{itemize}
        \item 科罗拉多大学的 \textbf{Gregory L. Plett} 将 EKF 引入锂电池管理 (BMS),解决了 SOC 估算难题。
              \footcite{Plett2004EKF2}
    \end{itemize}

\end{frame}

% \begin{frame}
%     \frametitle{现代发展:改进与替代}
%     为了解决线性化误差大的问题,学术界出现了新的方向:

%     \begin{table}
%         \centering
%         \tiny
%         \begin{tabular}{p{2.0cm}|p{4.0cm}|p{3.5cm}}
%             \toprule
%             \textbf{算法变体}                              & \textbf{核心原理} & \textbf{代表作} \\
%             \midrule
%             \textbf{UKF}                               &
%             使用 Sigma 点近似概率分布。                          &
%             Julier \& Uhlmann (1997)                                                  \\
%             \midrule
%             \textbf{ES-EKF}                            &
%             \textcolor{red}{主流}。在“误差状态”上做 EKF,解决四元数约束。 &
%             Joan Solà (2017)                                                          \\
%             \midrule
%             \textbf{MSC-KF}                            &
%             视觉里程计 (VIO) 专用,利用几何约束列观测方程。                &
%             Mourikis \& Roumeliotis (2007)                                            \\
%             \bottomrule
%         \end{tabular}
%     \end{table}

%     \footnotetext[1]{\fullcite{Julier1997UKF}}
%     \footnotetext[2]{\fullcite{Sola2017Quaternion}}
%     \footnotetext[3]{\fullcite{Mourikis2007MSCKF}}
% \end{frame}

% \begin{frame}
%     \frametitle{EKF 在各领域的具体实现路径总结}

%     \begin{columns}[T]
%         \column{0.5\textwidth}
%         \begin{alertblock}{微电子路径 (BMS/PLL)}
%             \begin{enumerate}
%                 \item \textbf{物理建模}:建立等效电路 (ECM) 或电化学模型。
%                 \item \textbf{参数辨识}:离线测试获取 R, C, OCV 表。
%                 \item \textbf{雅可比推导}:对查表函数进行数值微分。
%                 \item \textbf{嵌入式实现}:定点数运算优化,防止单片机算力过载。
%             \end{enumerate}
%         \end{alertblock}

%         \column{0.5\textwidth}
%         \begin{block}{信息工程路径 (SLAM/VIO)}
%             \begin{enumerate}
%                 \item \textbf{运动学建模}:李群与李代数 (Lie Group) 描述姿态。
%                 \item \textbf{误差定义}:$\delta \mathbf{x} = \mathbf{x}_{true} \ominus \mathbf{x}_{nominal}$。
%                 \item \textbf{观测模型}:重投影误差 (Reprojection Error)。
%                 \item \textbf{系统能观性分析}:确保零空间 (Null Space) 正确。
%             \end{enumerate}
%         \end{block}
%     \end{columns}
% \end{frame}

%================================================================
% 恢复原文的后续部分
%================================================================

\begin{frame}
    \frametitle{7. EKF 的优缺点分析}

    \begin{block}{优点}
        \begin{itemize}
            \item \textbf{事实标准}:目前导航系统(GPS/IMU 融合)、机器人定位 (SLAM) 的首选基础算法。
            \item \textbf{计算量适中}:比粒子滤波 (Particle Filter) 快得多。
        \end{itemize}
    \end{block}

    \begin{alertblock}{缺点}
        \begin{itemize}
            \item \textbf{发散风险}:如果线性化点选得不好(初始估计太差),泰勒展开误差会很大,导致滤波器发散。
            \item \textbf{繁琐的雅可比}:对于复杂的系统,计算雅可比矩阵非常痛苦且容易出错。
        \end{itemize}
    \end{alertblock}
\end{frame}

\begin{frame}
    \frametitle{8. 为什么我们需要 UKF?}

    当系统具有\textbf{强非线性}或雅可比矩阵\textbf{难以计算}时,EKF 会失效。

    \begin{columns}
        \column{0.5\textwidth}
        \begin{alertblock}{EKF 的痛点}
            \begin{enumerate}
                \item \textbf{精度丢失}:泰勒展开只保留了一阶项,对于剧烈变化的函数,高阶误差巨大。
                \item \textbf{计算难度大}:很多复杂物理模型(如流体、化学反应)根本写不出解析的雅可比矩阵 $\bm{F} = \frac{\partial f}{\partial x}$。
            \end{enumerate}
        \end{alertblock}

        \column{0.5\textwidth}
        \begin{block}{UKF 的哲学:无迹变换}
            \small
            \begin{itemize}
                \item \textbf{不线性化函数},而是去近似\textbf{概率分布}。
                \item \textbf{Sigma 点采样}:选取几个确定性的点,把它们扔进非线性函数里跑一遍。
                \item \textbf{加权重组}:根据变换后的点,统计出新的均值和协方差。
            \end{itemize}
        \end{block}
    \end{columns}

    \vspace{0.3cm}
    \centering
    \textit{“近似概率分布比近似任意非线性函数要容易得多。”} \\ —— Julier \& Uhlmann\footcite{Julier1997UKF}

\end{frame}

\begin{frame}
    \frametitle{总结:从 EKF 到 UKF 的演进逻辑}

    \begin{table}
        \centering
        \small
        \begin{tabular}{l|p{4.5cm}|p{4.5cm}}
            \toprule
            特性            & \textbf{EKF (扩展卡尔曼)} & \textbf{UKF (无迹卡尔曼)}         \\
            \midrule
            \textbf{核心思想} & \textbf{线性化}:用切线逼近曲线 & \textbf{采样}:用点集逼近分布          \\
            \midrule
            \textbf{数学工具} & 雅可比矩阵 (Jacobian)     & 无迹变换 (Unscented Transform)   \\
            \midrule
            \textbf{适用场景} & 轻度非线性 (如 GPS 定位)     & 强非线性、导数不存在 (如复杂化学反应、机动目标)    \\
            \midrule
            \textbf{工程代价} & 推导公式极难,计算极快          & 推导简单 (黑盒),计算量略大 ($2L+1$ 次运算) \\
            \bottomrule
        \end{tabular}
    \end{table}
\end{frame}
%================================================================

