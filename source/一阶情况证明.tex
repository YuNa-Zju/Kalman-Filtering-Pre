%================================================================
\subsection{公式推导——以一维为例}
%================================================================

\begin{frame}
    \frametitle{0. 设定场景:两个不靠谱的信源}
    假设我们要测量小车的\textbf{位置}。
    \begin{itemize}
        \item \textbf{信源 1(模型推算)}:
              \begin{itemize}
                  \item 值:$x_{k}^-$ (比如 100米)
                  \item 噪声:$w \sim N(0, P^-)$ (方差为 $P^-$)
              \end{itemize}
        \item \textbf{信源 2(传感器测量)}:
              \begin{itemize}
                  \item 值:$z_k$ (比如 102米)
                  \item 噪声:$v \sim N(0, R)$ (方差为 $R$)
              \end{itemize}
    \end{itemize}
    \vspace{0.3cm}
    \textbf{目标}:找到一个最优估计 $\hat{x}_k$,让它的\textbf{误差方差}最小。
\end{frame}

\begin{frame}[allowframebreaks]
    \frametitle{1. 构建融合方程}
    根据直觉,我们采用\textbf{加权平均}的方式来融合:
    $$ \hat{x}_k = (1-K)x_{k}^- + K z_k $$

    \begin{block}{💡 为什么系数是这样分配的?}
        \footnotesize
        这源于\textbf{“修正”}的思想:$\hat{x}_k = x_{k}^- + K(z_k - x_{k}^-)$。
        \begin{itemize}
            \item 我们以\textbf{预测} $x_{k}^-$ 为基准。
            \item $K$ 代表我们多大程度上相信\textbf{观测} $z_k$ 来修正这个基准。
            \item 展开后即得:$(1-K)x_{k}^- + K z_k$。
        \end{itemize}
    \end{block}

    \textbf{代数变形}:
    假设真实值是 $x_{true}$。
    \begin{itemize}
        \item 推算值 $x_{k}^- = x_{true} + w$
        \item 观测值 $z_k = x_{true} + v$
    \end{itemize}

    代入融合方程:
    \begin{align*}
        \hat{x}_k & = (1-K)(x_{true} + w) + K(x_{true} + v)                                       \\
                  & = (1-K)x_{true} + (1-K)w + K x_{true} + K v                                   \\
                  & = \underbrace{x_{true}}_{\text{真值}} + \underbrace{(1-K)w + K v}_{\text{综合误差}}
    \end{align*}
\end{frame}

\begin{frame}
    \frametitle{1.5 深度思考:为什么不用几何/平方平均?}
    有同学可能会问:为什么要用线性的 $(1-K)a + Kb$,用更高级的公式不行吗?

    \vspace{0.2cm}

    \begin{columns}
        \column{0.48\textwidth}
        \begin{alertblock}{方案 A:几何平均 $\sqrt{x \cdot z}$}
            \textbf{致命缺陷:负数地狱}
            \begin{itemize}
                \item 状态可能是负数(如倒车速度为 -5m/s)。
                \item 根号下出现负数会导致系统崩溃(变成虚数)。
            \end{itemize}
        \end{alertblock}

        \column{0.48\textwidth}
        \begin{alertblock}{方案 B:平方平均 $\sqrt{\frac{x^2+z^2}{2}}$}
            \textbf{致命缺陷:引入偏差}
            \begin{itemize}
                \item 噪声平方后恒为正。
                \item 这会导致估计值\textbf{永远偏大},不再是“无偏估计”。
            \end{itemize}
        \end{alertblock}
    \end{columns}

    \vspace{0.4cm}
    \centering
    \textbf{结论}:只有\textbf{线性组合}能保持高斯分布特性,并保证平均误差为 0。
\end{frame}

\begin{frame}
    \frametitle{2. 计算综合方差}
    我们的目标是让\textbf{综合误差}的方差最小。

    令总误差为 $e = (1-K)w + K v$。

    因为$w$ 和 $v$ 是独立的,于是可以利用方差性质,得到融合后的总方差 $P_k$ 为:
    \begin{equation}
        P_k(K) = \underbrace{(1-K)^2 P^-}_{\text{推算部分的贡献}} + \underbrace{K^2 R}_{\text{测量部分的贡献}}
    \end{equation}

    \small \centering 这就是我们要优化的目标函数,它是关于 $K$ 的开口向上的抛物线。
\end{frame}

\begin{frame}
    \frametitle{3. 求极值}
    为了找到方差最小的$K$,我们对 $K$ 求导并令其为 0。

    $$ \frac{d P_k}{d K} = \frac{d}{d K} [ (1-K)^2 P^- + K^2 R ] = 0 $$

    整理上式,把含 $K$ 的项放在左边,常数项放在右边:

    $$ K(P^- + R) = P^- $$
    \begin{alertblock}{最优卡尔曼增益 (Kalman Gain)}
        \centering \Large
        $$ K = \frac{P^-}{P^- + R} $$
    \end{alertblock}
\end{frame}