%================================================================
\section{前沿探索}
%================================================================

% --- 第一部分:基于顶级综述的算法演进 (Khodarahmi et al., 2023) ---

\begin{frame}
    \frametitle{1. 算法演进全景 (Based on Top Review)}
    \small
    卡尔曼滤波已发展为庞大的家族。针对不同场景,主流算法的特性对比如下\footcite{khodarahmi_review_2023}:

    % 表格:主流算法对比
    \begin{table}
        \centering
        \renewcommand{\arraystretch}{1.2}
        \resizebox{0.95\textwidth}{!}{
            \begin{tabular}{l l l l}
                \toprule
                \textbf{算法}       & \textbf{核心机制}      & \textbf{优势 (Pros)}   & \textbf{适用场景} \\
                \midrule
                \textbf{Basic KF} & 线性递归最小二乘           & 计算极快,线性最优            & 卫星导航,稳态控制     \\
                \textbf{EKF}      & 泰勒展开线性化 ($\bm{J}$) & 工业标准,适用性广            & 机器人定位 (SLAM)  \\
                \textbf{UKF}      & 无迹变换 (Sigma点)      & 无需导数,精度更高            & 复杂非线性系统       \\
                \textbf{IMM}      & 多模型概率交互            & \textbf{搞定机动目标},平滑切换 & 导弹/无人机追踪      \\
                \bottomrule
            \end{tabular}
        }
    \end{table}
\end{frame}

% --- 第二部分:KF 与神经网络的结合 (Feng et al., 2023) ---

\begin{frame}
    \frametitle{2. 突破瓶颈:混合模型 (Hybrid Models) 的诞生}

    \begin{columns}
        \column{0.6\textwidth}
        % 左侧:放流程图(占位符)
        \centering
        \begin{figure}
            % 这里的 hybrid.png 就是您那张流程图
            % keepaspectratio 保持长宽比,width 设置为栏宽的 95%
            \includegraphics[width=0.95\textwidth, keepaspectratio]{hybrid.png}
            \caption{状态估计技术融合路线图\footcite{feng_review_2023}}
        \end{figure}

        \column{0.4\textwidth}
        % 右侧:文字解说
        \textbf{为何要融合?}
        \begin{itemize}
            \item \textbf{传统 KF}:依赖精确物理模型,难以处理未知环境。
            \item \textbf{神经网络 (NN)}:强大的数据拟合能力,但缺乏物理约束。
        \end{itemize}

        \vspace{0.3cm}
        \textbf{融合的三种模式:}
        \begin{enumerate}
            \small
            \item \textbf{串联}:互为预处理/后处理。
            \item \textbf{训练}:KF 优化 NN 权重。
            \item \textbf{辅助 (Aided)}:\textcolor{red}{\textbf{NN 实时估计 KF 参数 ($\bm{Q}, \bm{R}$)}}。
        \end{enumerate}
    \end{columns}
\end{frame}
\begin{frame}
    \frametitle{深度思考:卡尔曼滤波 vs 机器学习}
    \begin{columns}
        % 左栏:卡尔曼滤波 (Model-Based)
        \column{0.48\textwidth}
        \begin{block}{卡尔曼滤波 (KF)}
            \centering \textbf{“理性的物理学家”}
            \begin{itemize}
                \item \textbf{驱动核心}:物理模型 ($\bm{F}, \bm{H}$) + 概率统计。
                \item \textbf{透明度}:\textbf{白盒}。每一步都有明确物理含义,可解释性强。
                \item \textbf{优势}:小样本即可工作,不仅给结果,还给\textbf{置信度} ($\bm{P}$)。
                \item \textbf{劣势}:模型必须已知且准确。
            \end{itemize}
        \end{block}

        % 右栏:机器学习 (Data-Driven)
        \column{0.48\textwidth}
        \begin{exampleblock}{机器学习 (ML/NN)}
            \centering \textbf{“经验丰富的工匠”}
            \begin{itemize}
                \item \textbf{驱动核心}:海量数据 + 拟合映射。
                \item \textbf{透明度}:\textbf{黑盒}。内部权重难以解释。
                \item \textbf{优势}:能拟合极其复杂的非线性关系,无需懂物理机理。
                \item \textbf{劣势}:数据饥渴,对未见过的场景泛化能力弱。
            \end{itemize}
        \end{exampleblock}
    \end{columns}

    \vspace{0.4cm}
    \centering
    \textbf{融合的哲学:}
    \fcolorbox{blue}{white}{用 KF 的逻辑框架约束 ML 的发散,用 ML 的拟合能力弥补 KF 的模型缺陷。}
\end{frame}
% --- 第三部分:锂电池 SOC 估计应用 ---

% \begin{frame}
%     \frametitle{3. 落地应用:锂电池 SOC 状态估计}
%     \small
%     \textbf{背景:} SOC (State of Charge) 是电池管理系统的核心。估算错误会导致过充/过放,造成永久性损伤。

%     \vspace{0.3cm}
%     \textbf{核心技术栈 (Tech Stack):}
%     \begin{columns}[t]
%         \column{0.48\textwidth}
%         \begin{block}{1. 非线性滤波 (NLKFs)}
%             用于处理电池高度非线性的电压特性:
%             \begin{itemize}
%                 \item \textbf{EKF / AEKF} (自适应扩展卡尔曼)
%                 \item \textbf{UKF / AUKF} (自适应无迹卡尔曼)
%             \end{itemize}
%         \end{block}

%         \column{0.48\textwidth}
%         \begin{block}{2. 在线参数辨识}
%             实时更新电池模型参数(如内阻):
%             \begin{itemize}
%                 \item \textbf{RLS} (递归最小二乘法)
%                 \item \textbf{PRBM} (多项式回归模型)
%             \end{itemize}
%         \end{block}
%     \end{columns}

%     \vspace{0.4cm}
%     \textbf{实验结论:}
%     在宽温域 (-5\textcelsius $\sim$ 45\textcelsius) 测试中,以下组合表现最佳:
%     \begin{center}
%         \fcolorbox{red}{white}{\textbf{PRBM-AUKF} \quad \& \quad \textbf{RLS-AUKF}}
%     \end{center}
%     \footnotesize \textit{* AUKF (Adaptive UKF) 展现了比传统 EKF 更高的精度和鲁棒性。\footcite{hossain_kalman_2022}}
% \end{frame}


\begin{frame}
    \frametitle{3.1 落地应用Ⅰ:信号处理领域}
    
    \begin{columns}[c] % [c] 参数使左右两列内容垂直居中对齐
        % 左侧:放置图片
        \column{0.5\textwidth}
        \begin{figure}
            \centering
            % 建议根据侧边布局调整宽度,0.9\columnwidth 比较合适
            \includegraphics[width=0.9\columnwidth]{figures/application.png}
            % \caption{卡尔曼滤波典型应用场景}
        \end{figure}

        % 右侧:放置信号处理应用文字
        \column{0.5\textwidth}
        \textbf{信号处理主要应用} \\
        \scriptsize (Key Signal Processing Applications)
        
        \vspace{0.3cm} % 增加一点间距
        
        \begin{itemize}
            \item \textbf{时间序列模型识别}:如自回归(AR)、滑动平均(MA)等。
            \item \textbf{移动目标定位与跟踪}。
            \item \textbf{去噪、信号增强以及反卷积}。
            \item \textbf{无线传感器网络与分布式估计}。
            \item \textbf{生物医学应用}。
        \end{itemize}
    \end{columns}
\end{frame}

\begin{frame}
    \frametitle{3.2  落地应用Ⅱ:集成电路(IC)热管理}
    \begin{tikzpicture}[remember picture,overlay]
    \node[anchor=north east, xshift=-0.5cm, yshift=-1.2cm] at (current page.north east) {
        \includegraphics[width=3.5cm]{figures/thermal_management.jpg}
    };
\end{tikzpicture}
    \small
    \textbf{背景:} 随着高性能芯片(如 CPU、GPU、AI 加速器)功率密度增加,散热
    \par
    成为巨大挑战。

    \vspace{0.2cm}
    \textbf{核心技术栈 (Tech Stack):}
    \begin{columns}[t]
        \column{0.48\textwidth}
        \begin{block}{1. 热点估计:}
            芯片内部无法布设无限多的温度传感器。卡尔曼滤波利用有限的片上温度传感器数据,结合芯片的热传导数学模型,实时估计整个硅片表面的温度分布情况(热图)。
        \end{block}

        \column{0.48\textwidth}
        \begin{block}{2. 动态频率调整}
            基于卡尔曼滤波预测的温度趋势,芯片可以更平滑地调整工作频率,避免因温度突增导致的系统降频或损坏。
        \end{block}
    \end{columns}

    \vspace{0.3cm}
    \textbf{为什么微电子领域青睐卡尔曼滤波?}
    % 在宽温域 (-5\textcelsius $\sim$ 45\textcelsius) 测试中,以下组合表现最佳:
    % \begin{center}
    %     \fcolorbox{red}{white}{\textbf{PRBM-AUKF} \quad \& \quad \textbf{RLS-AUKF}}
    % \end{center}
    % \footnotesize \textit{* AUKF (Adaptive UKF) 展现了比传统 EKF 更高的精度和鲁棒性。\footcite{hossain_kalman_2022}}
    \begin{itemize}
        \item \textbf{硬件资源友好:}卡尔曼滤波的递归特性意味着它不需要存储海量的历史数据,非常适合内存和计算资源有限的微控制器(MCU)或嵌入式逻辑。
        \item \textbf{抗电磁干扰:}微电子环境充满了高频开关噪声,卡尔曼滤波能通过设置测量噪声矩阵($R$),有效地屏蔽电路层面的随机干扰。
        \item \textbf{实时响应:}它的计算延迟极低,能够满足微秒级的实时控制需求。
    \end{itemize}
\end{frame}

\begin{frame}
    \frametitle{3.3 落地应用Ⅲ:微纳制造过程控制}
    \small
    \textbf{背景:} 在半导体制造工艺(如光刻、化学机械抛光 CMP)中,传感器数据的精密反馈至关重要。
    % \vspace{0.1cm}

    \begin{columns}[t]
        \column{0.45\textwidth}
        % \textbf{挑战:}
        %     \begin{itemize}
        %         \item \textbf{测量噪声大:} 研磨液、高速旋转、高压导致传感器信号剧烈波动。
        %         \item \textbf{抛光速率时变:} 抛光垫磨损、研磨液变化导致速率漂移。
        %         \item \textbf{终点检测难:} 传统方法精度不足,易导致过抛或欠抛。
        %     \end{itemize}

        % \vspace{0.1cm}
        \textbf{核心技术:利用卡尔曼滤波进行状态估计}
            \begin{itemize}
                \item \textbf{状态向量 $x_k = \begin{bmatrix} h_k \\ r_k \end{bmatrix}$:} 估计真实薄膜厚度 ($h_k$) 和实时抛光速率 ($r_k$)。
                \item \textbf{动态模型:}
                    \begin{itemize}
                        \item $h_k = h_{k-1} - r_{k-1} \cdot \Delta t + w_{h}$ (厚度变化)
                        \item $r_k = r_{k-1} + w_{r}$ (速率变化)
                    \end{itemize}
                \item \textbf{观测模型 $z_k = h_k + v_k$:} 传感器测量厚度(含严重噪声 $v_k$)。
            \end{itemize}

        \column{0.55\textwidth} % 右侧用于放置图片
            \begin{figure}
                \centering
                % 请将图片保存为 "cmp_kalman_filter_示意图.png" 并与.tex文件放在同一目录
                \includegraphics[width=0.9\textwidth]{figures/cmp_kalman_filter_示意图.png}
                \label{fig:cmp_kalman}
            \end{figure}
            
            \vspace{-0.35cm} % 调整图片下方的空间
            \textbf{卡尔曼滤波的优势:}
            \begin{itemize}
                \item \textbf{实时去噪与平滑:} 过滤传感器信号的剧烈波动,输出平滑准确的厚度曲线。
                \item \textbf{动态速率估计:} 实时修正抛光速率,适应工艺参数变化。
                \item \textbf{精准终点预测:} 基于滤波后的厚度及预测,实现纳米级停机精度,避免过抛/欠抛。
                \item \textbf{无延迟:} 相比传统低通滤波,实现实时去噪而不引入时间延迟。
            \end{itemize}
    \end{columns}

\end{frame}