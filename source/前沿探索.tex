%================================================================
\section[前沿]{前沿探索}
%================================================================

% --- 第一部分:基于顶级综述的算法演进 (Khodarahmi et al., 2023) ---
\subsection{算法演进全景}
\begin{frame}
    \frametitle{1. 算法演进全景}
    \small
    卡尔曼滤波已发展为庞大的家族。针对不同场景,主流算法的特性对比如下\footcite{khodarahmi_review_2023}:

    % 表格:主流算法对比
    \begin{table}
        \centering
        \renewcommand{\arraystretch}{1.2}
        \resizebox{0.95\textwidth}{!}{
            \begin{tabular}{l l l l}
                \toprule
                \textbf{算法}       & \textbf{核心机制}      & \textbf{优势 (Pros)}   & \textbf{适用场景} \\
                \midrule
                \textbf{Basic KF} & 线性递归最小二乘           & 计算极快,线性最优            & 卫星导航,稳态控制     \\
                \textbf{EKF}      & 泰勒展开线性化 ($\bm{J}$) & 工业标准,适用性广            & 机器人定位 (SLAM)  \\
                \textbf{UKF}      & 无迹变换 (Sigma点)      & 无需导数,精度更高            & 复杂非线性系统       \\
                \textbf{IMM}      & 多模型概率交互            & \textbf{搞定机动目标},平滑切换 & 导弹/无人机追踪      \\
                \bottomrule
            \end{tabular}
        }
    \end{table}
\end{frame}

% --- 第二部分:KF 与神经网络的结合 (Feng et al., 2023) ---
\subsection{卡尔曼滤波遇上神经网络}
\begin{frame}
    \frametitle{2. 突破瓶颈:混合模型的诞生}

    \begin{columns}
        \column{0.6\textwidth}
        % 左侧:放流程图(占位符)
        \centering
        \begin{figure}
            % 这里的 hybrid.png 就是您那张流程图
            % keepaspectratio 保持长宽比,width 设置为栏宽的 95%
            \includegraphics[width=0.95\textwidth, keepaspectratio]{hybrid.png}
            \caption{状态估计技术融合路线图\footcite{feng_review_2023}}
        \end{figure}

        \column{0.4\textwidth}
        % 右侧:文字解说
        \textbf{为何要融合?}
        \begin{itemize}
            \item \textbf{传统 KF}:依赖精确物理模型,难以处理未知环境。
            \item \textbf{神经网络 (NN)}:强大的数据拟合能力,但缺乏物理约束。
        \end{itemize}

        \vspace{0.3cm}
        \textbf{融合的三种模式:}
        \begin{enumerate}
            \small
            \item \textbf{串联}:互为预处理/后处理。
            \item \textbf{训练}:KF 优化 NN 权重。
            \item \textbf{辅助 (Aided)}:\textcolor{red}{\textbf{NN 实时估计 KF 参数 ($\bm{Q}, \bm{R}$)}}。
        \end{enumerate}
    \end{columns}
\end{frame}
\begin{frame}
    \frametitle{深度思考:卡尔曼滤波 vs 机器学习}
    \begin{columns}
        % 左栏:卡尔曼滤波 (Model-Based)
        \column{0.48\textwidth}
        \begin{block}{卡尔曼滤波 (KF)}
            \centering \textbf{“理性的物理学家”}
            \begin{itemize}
                \item \textbf{驱动核心}:物理模型 ($\bm{F}, \bm{H}$) + 概率统计。
                \item \textbf{透明度}:\textbf{白盒}。每一步都有明确物理含义,可解释性强。
                \item \textbf{优势}:小样本即可工作,不仅给结果,还给\textbf{置信度} ($\bm{P}$)。
                \item \textbf{劣势}:模型必须已知且准确。
            \end{itemize}
        \end{block}

        % 右栏:机器学习 (Data-Driven)
        \column{0.48\textwidth}
        \begin{exampleblock}{机器学习 (ML/NN)}
            \centering \textbf{“经验丰富的工匠”}
            \begin{itemize}
                \item \textbf{驱动核心}:海量数据 + 拟合映射。
                \item \textbf{透明度}:\textbf{黑盒}。内部权重难以解释。
                \item \textbf{优势}:能拟合极其复杂的非线性关系,无需懂物理机理。
                \item \textbf{劣势}:数据饥渴,对未见过的场景泛化能力弱。
            \end{itemize}
        \end{exampleblock}
    \end{columns}

    \vspace{0.4cm}
    \centering
    \textbf{融合的哲学:}
    \fcolorbox{blue}{white}{用 KF 的逻辑框架约束 ML 的发散,用 ML 的拟合能力弥补 KF 的模型缺陷。}
\end{frame}

\subsection{前沿案例}

\begin{frame}
    \frametitle{雷达追踪中的“奇异点”}

    在雷达追踪中,如果飞机做\textbf{协同转弯运动 (CTRV)},状态向量包含转弯率 $\omega$:
    $$ \bm{x} = [p_x, p_y, v, \psi, \omega]^T $$

    \textbf{状态转移方程:}
    \begin{equation}
        p_{x, k+1} = p_{x, k} + \frac{v_k}{\omega_k} (\sin(\psi_k + \omega_k \Delta t) - \sin(\psi_k))
    \end{equation}

    \begin{alertblock}{EKF 在此处的致命缺陷}
        当我们尝试求雅可比矩阵 $\frac{\partial f}{\partial \omega}$ 时,分母含有 $\omega$。
        \begin{itemize}
            \item 当飞机\textbf{直线飞行}时,$\omega \to 0$。
            \item 此时雅可比矩阵中会出现 $\frac{0}{0}$ 型的\textbf{除零奇点},导致 EKF 数值爆炸。
        \end{itemize}
    \end{alertblock}

    \textbf{UKF 优势:} 不需要求导,直接代入数值计算,完美避开数学奇点。\footcite{Li2003Survey}
\end{frame}

\begin{frame}
    \frametitle{基础阶段:基于二阶 RC 模型的 SOC 估算}

    \begin{columns}
        \column{0.45\textwidth}
        \centering
        % 图形保持不变,视觉核心
        \begin{tikzpicture}[scale=0.65, transform shape]
            \draw (0,2) to[battery1, l=$V_{OCV}$] (0,0);
            \draw (0,2) to[R, l=$R_0$, i=$I$] (2,2);
            \draw (2,2) -- (3,2);
            \draw (3,2) to[R, l=$R_p$] (3,0);
            \draw (3,2) -- (4.5,2) to[C, l=$C_p$] (4.5,0) -- (3,0);
            \draw (3,0) -- (0,0);
            \draw (4.5,2) -- (6,2);
            \draw (4.5,0) -- (6,0);
            \draw[->] (6,1.8) -- (6,0.2) node[midway, right] {$V_{term}$};
        \end{tikzpicture}

        \column{0.55\textwidth}
        \textbf{1. 物理建模:}
        利用二阶 RC 电路模拟电池极化效应。

        \vspace{0.2cm}
        \textbf{2. 状态定义:}
        $$ \bm{x} = [SOC, V_{p}]^T $$

        \textbf{3. 核心挑战:}
        $V_{OCV}$ 与 $SOC$ 呈非线性关系(查表),导致无法直接使用线性卡尔曼滤波 (KF)。
    \end{columns}

    \vspace{0.3cm}
    \begin{block}{工业界标准解法:EKF (扩展卡尔曼滤波)}
        由于 $V_{OCV}$ 曲线相对平滑,我们通过\textbf{一阶泰勒展开}(求导)即可获得良好的线性化近似。
        $$ \bm{H}_k = \left[ \frac{\partial V_{OCV}}{\partial SOC}, -1 \right] $$
        此时,EKF 能够有效修正电流积分带来的累积误差。
    \end{block}
\end{frame}

\begin{frame}
    \frametitle{进阶挑战:电池老化带来的“模型失配”}

    \textbf{现实困境:}
    EKF 准确的前提是\textbf{模型参数 ($R_0, R_p, C_p$) 是已知且准确的}。
    \begin{itemize}
        \item 然而,随着电池老化和温度变化,内阻 $R_0$ 可能增加 100\%,电容 $C_p$ 也会漂移。
        \item \alert{后果:} 如果坚持使用固定参数模型,EKF 会强行把模型误差“归咎”于 SOC,导致 SOC 估算严重偏离。
    \end{itemize}

    \vspace{0.5cm}
    \textbf{解决方案:联合估算}
    将参数也视为状态的一部分,构建\textbf{增强状态向量}:
    $$ \bm{X}_{aug} = [\underbrace{SOC, V_p}_{\text{原状态}}, \underbrace{R_0, R_p, C_p}_{\text{漂移参数}}]^T $$
    我们要同时估算这 5 个变量。
\end{frame}

% --- 第三阶段:EKF 的局限与 UKF 的登场 ---
\begin{frame}
    \frametitle{方法论对比:为何参数辨识需要 UKF?}

    面对增强后的高维非线性系统,EKF 显得力不从心,而 UKF 展现出降维打击的优势:

    \begin{columns}[t]
        \column{0.5\textwidth}
        \begin{alertblock}{EKF 的死穴:求导困难}
            \footnotesize
            EKF 需要计算雅可比矩阵 $\frac{\partial h}{\partial \bm{X}}$。
            \begin{itemize}
                \item 对 $R_0$ 求导很简单。
                \item \textbf{但对 $C_p$ 求导极其痛苦}:$C_p$ 位于离散化方程的指数项 $e^{-\Delta t / (R_p C_p)}$ 中,且涉及递归。
            \end{itemize}
            \centering
            $\Downarrow$ \\
            推导繁琐,极易出错,数值不稳定。
        \end{alertblock}

        \column{0.5\textwidth}
        \begin{exampleblock}{UKF 的优势:黑盒思维}
            \footnotesize
            UKF不需要计算雅可比矩阵。
            \begin{itemize}
                \item \textbf{直接仿真}:根据当前参数 $R, C$ 生成一组 Sigma 点,直接代入电路公式算出预测电压。
                \item \textbf{自动拟合}:算法自动调整参数分布以逼近观测电压。
            \end{itemize}
            \centering
            $\Downarrow$ \\
            \textbf{无需推导公式,仅需物理模型代码。}
        \end{exampleblock}
    \end{columns}

\end{frame}