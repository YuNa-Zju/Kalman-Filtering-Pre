%================================================================
\section[建模]{数学模型构建}
%================================================================
\begin{frame}
\frametitle{用向量描述系统状态}

\begin{itemize}
    \item 在同一时刻,系统通常需要多个物理量共同刻画,这些量彼此相关、共同变化。

    \vspace{0.25cm}
    \item 一个自然的做法是:将它们组合为一个向量(如 [位置, 速度]$^T$)
    \[
        \bm{x}_k =
        \begin{bmatrix}
            x_k^{(1)} & x_k^{(2)} & \cdots & x_k^{(n)}
        \end{bmatrix}^\mathsf{T}
    \]

    \vspace{0.25cm}
    \item 这种表示的优势:
          \begin{itemize}
              \item 在一个对象中同时保留多个物理量
              \item 物理量之间的\textbf{线性关系}可由线性映射刻画
              \item 便于统一描述随时间变化的整体状态
          \end{itemize}
    \item 与系统状态类似地,还可以用向量表示观测值、噪声、传入系统的控制量;用矩阵表示由观测值到系统状态的换算。
\end{itemize}

\vspace{0.25cm}
\centering
\textbf{后续方程只涉及对向量、矩阵的变化描述}
\end{frame}



\begin{frame}
    \frametitle{1. 状态方程:系统的物理特性}
    % 说明:这是建立在上一时刻基础上的“先验估计”
    \begin{equation}
        \bm{x}_k = \underbrace{\bm{F} \bm{x}_{k-1}}_{\text{状态转移}} + \underbrace{\bm{B} \bm{u}_k}_{\text{控制输入}} + \underbrace{\bm{w}_k}_{\text{过程噪声}}
    \end{equation}

    \vspace{0.5cm}

    \begin{itemize}
        \setlength{\itemsep}{12pt} % 增加行间距,显得更从容
        \item \textbf{状态转移项 $\bm{F}\bm{x}_{k-1}$}
              \newline
              描述系统在无外界干预下的自然变化规律。$\bm{F}$ 为状态转移矩阵,例如在匀速模型中,它将上一刻的位置和速度映射到当前时刻。

        \item \textbf{控制输入项 $\bm{B}\bm{u}_k$}
              \newline
              描述外部主动控制对系统的影响。其中 $\bm{u}_k$ 是已知的控制量(如油门开度、方向盘转角),$\bm{B}$ 为控制矩阵。
              \newline {\footnotesize{\textit{注:在纯跟踪问题中,若无主动控制,此项为零。}}}

        \item \textbf{过程噪声 $\bm{w}_k$}
              \newline
              代表物理模型与真实世界之间的偏差,包含风阻、摩擦等难以建模的随机扰动。
    \end{itemize}
\end{frame}

\begin{frame}
    \frametitle{2. 观测方程:传感器的特性}
    % 说明:这是建立在传感器读数基础上的描述
    \begin{equation}
        \bm{z}_k = \underbrace{\bm{H} \bm{x}_k}_{\text{观测映射}} + \underbrace{\bm{v}_k}_{\text{测量噪声}}
    \end{equation}

    \vspace{0.5cm}

    \begin{itemize}
        \setlength{\itemsep}{12pt}
        \item \textbf{观测向量$z_k$}
              \newline $k$时刻所有传感器读数按一定顺序排列的列向量。
        \item \textbf{观测映射项 $\bm{H}\bm{x}_k$}
              \newline
              描述系统状态通过何种映射得到传感器读数。
              \newline
              例如:$\bm{H} = [1, 0]$ 可将二维状态向量投影为一维观测值。在$\bm{x}_k = [\text{位移},\text{速度}]^T$的情形下,这样的$\bm{H}$对应了一个位置传感器。

        \item \textbf{测量噪声 $\bm{v}_k$}
              \newline
              源于传感器自身的物理特性,如电子热噪声、量化误差或环境干扰。
    \end{itemize}
\end{frame}

\begin{frame}
    \frametitle{符号定义与噪声假设}
    状态量的维数:$n$;观测量的维数:$m$;控制量的维数:$l$。
    \begin{table}
        \centering
        \renewcommand{\arraystretch}{1.3} % 增加表格行高
        \begin{tabular}{c l c}
            \toprule
            符号         & 物理含义                       & 维度           \\
            \midrule
            $\bm{x}_k$ & $k$时刻的状态向量 & $n \times 1$ \\
            $\bm{F}$   & 状态转移矩阵                     & $n \times n$ \\
            $\bm{B}$   & 控制输入矩阵                     & $n \times l$ \\
            $\bm{z}_k$ & $k$时刻的观测向量          & $m \times 1$ \\
            $\bm{H}$   & 观测矩阵                       & $m \times n$ \\
            \bottomrule
        \end{tabular}
    \end{table}

    \vspace{0.4cm}

    假设过程噪声和测量噪声都服从均值为$0$的正态分布。
    $$
        \bm{w}_k \sim N(0, \bm{Q}), \quad \bm{v}_k \sim N(0, \bm{R})
    $$
    其中 $\bm{Q}$ 为过程噪声协方差矩阵,$\bm{R}$ 为测量噪声协方差矩阵。
\end{frame}