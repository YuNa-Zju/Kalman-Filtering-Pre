%================================================================
\section[核心算法]{核心算法流程}
%================================================================

\subsection{算法总览}
\begin{frame}
    \frametitle{算法总览}
    卡尔曼滤波是不断预测——更新的过程。每经过一段时间观测一次,就执行一次循环。

    \vspace{0.8cm}

    \begin{center}
        \begin{tikzpicture}[node distance=2cm, auto, >=latex]
            % 定义样式
            \tikzstyle{block} = [rectangle, draw, fill=blue!10,
            text width=10em, text centered, rounded corners, minimum height=3em]
            \tikzstyle{line} = [draw, ->, thick]

            % 节点
            \node [block] (pred) {1. 先验估计 \\ \footnotesize 根据之前数据计算先验状态和协方差};
            \node [block, below of=pred] (gain) {2. 计算增益\\ \footnotesize 计算卡尔曼增益 $\bm{K}_k$};
            \node [block, below of=gain] (corr) {3. 观测更新\\ \footnotesize 用观测数据校正先验估计};

            % 连线
            \path [line] (pred) -- (gain);
            \path [line] (gain) -- (corr);
            % 画一个循环箭头回去
            \path [line] (corr.east) -- ++(1,0) |- (pred.east);
        \end{tikzpicture}
    \end{center}
\end{frame}

\subsection{预测}
\begin{frame}
    \frametitle{第一阶段:时间更新}
    在不涉及传感器数据的情况下,基于上一时刻的状态与物理模型进行先验估计。

    \vspace{0.5cm}

    \begin{block}{1. 状态先验估计}
        \begin{equation}
            \hat{\bm{x}}_k^- = \bm{F} \hat{\bm{x}}_{k-1} + \bm{B} \bm{u}_k
        \end{equation}
        系统根据状态转移矩阵 $\bm{F}$ 和控制输入 $\bm{B}\bm{u}_k$ 推演当前时刻的理论状态。
    \end{block}

    \vspace{0.3cm}

    \begin{block}{2. 协方差先验估计}
        \begin{equation}
            \bm{P}_k^- = \bm{F} \bm{P}_{k-1} \bm{F}^T + \bm{Q}
        \end{equation}
        $\bm{P}_k^-$是真实值和预测值的先验误差协方差矩阵,$\bm{P}_k$是真实值和预测值的后验误差协方差矩阵,
        误差随时间传递与扩散。
    \end{block}
\end{frame}

\begin{frame}
    \frametitle{卡尔曼增益}

    卡尔曼增益 $\bm{K}_k$ 是最小均方误差准则下的最优加权因子。

    \vspace{0.5cm}
    \begin{columns}
        \column{0.5\textwidth}
        \textbf{一维形式的直觉:}
        此时公式可简化为:
        \begin{equation*}
            K_k = \frac{P_k^-}{P_k^- + R}
        \end{equation*}

        \column{0.5\textwidth}
        \textbf{物理含义:}
        $$ \text{增益} = \frac{\text{预测不确定性}}{\text{预测不确定性} + \text{观测不确定性}} $$
    \end{columns}

    \vspace{0.5cm}
    \begin{itemize}
        \item 分子代表模型预测的误差方差。
        \item 分母代表系统总的误差方差。
        \item $\bm{K}_k$ 动态衡量了我们应该多大程度上信任观测值。
    \end{itemize}
\end{frame}


% --- 核心:完整公式 ---
\begin{frame}
    \frametitle{矩阵形式的完整表达}
    在多维系统中,我们需要处理状态空间与观测空间的维度差异。

    \begin{block}{3. 计算卡尔曼增益}
        \begin{equation}
            \bm{K}_k = \bm{P}_k^- \bm{H}^T (\bm{H} \bm{P}_k^- \bm{H}^T + \bm{R})^{-1}
        \end{equation}
        项 $(\bm{H} \bm{P}_k^- \bm{H}^T + \bm{R})^{-1}$ 充当了归一化因子的角色。
    \end{block}
\end{frame}

\subsection{更新}
\begin{frame}
    \frametitle{第二阶段:状态更新}
    利用计算出的增益,融合观测数据,得到最优的后验估计。

    \vspace{0.3cm}

    \begin{block}{4. 状态后验更新}
        \begin{equation}
            \hat{\bm{x}}_k = \hat{\bm{x}}_k^- + \bm{K}_k (\bm{z}_k - \bm{H} \hat{\bm{x}}_k^-)
        \end{equation}
        \textbf{后验估计 = 先验预测 + 增益 $\times$ 新息}。
        \newline
        其中 $(\bm{z}_k - \bm{H} \hat{\bm{x}}_k^-)$ 被称为新息 (Innovation),代表观测值与预测值的偏差。
    \end{block}

    \vspace{0.3cm}

    \begin{block}{5. 协方差后验更新}
        \begin{equation}
            \bm{P}_k = (\bm{I} - \bm{K}_k \bm{H}) \bm{P}_k^-
        \end{equation}
        由于引入了观测信息,系统的不确定性得以降低 ($\bm{P}_k < \bm{P}_k^-$),为下一时刻的计算做准备。
    \end{block}
\end{frame}

\begin{frame}
    \frametitle{极限情形下的一些直觉}
    % 通过分析 $\bm{K}_k$ 在极限情况下的表现,可以理解其融合逻辑:

    \vspace{0.3cm}

    \textbf{情形一:传感器极其精准} ($\bm{R} \to 0$)
    \begin{itemize}
        \item $K_k \to 1$
        \item $\hat{x}_k = \hat{x}_k^- + 1 \cdot (z_k - \hat{x}_k^-) = z_k$
        \item 滤波器完全信任观测值,忽略模型预测。
    \end{itemize}

    \vspace{0.4cm}

    \textbf{情形二:模型极其精准或传感器故障} ($\bm{P}_k^- \to 0$ 或 $\bm{R} \to \infty$)
    \begin{itemize}
        \item $K_k \to 0$
        \item $\hat{x}_k = \hat{x}_k^- + 0 = \hat{x}_k^-$
        \item 滤波器完全信任模型预测,忽略观测数据。
    \end{itemize}

    \vspace{0.4cm}
    \centering
    \textit{结论:卡尔曼滤波总是自动偏向于“方差更小”的一方。}
\end{frame}
