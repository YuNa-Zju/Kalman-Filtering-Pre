%================================================================
\section{背景引入}
%================================================================

% --- 第一页:直观痛点(增加代入感) ---
\begin{frame}
	\frametitle{直觉挑战:蒙眼走路的困境}

	\begin{columns}
		\column{0.6\textwidth}
		想象你被蒙住双眼,试图沿着直线走 100 米:
		\begin{itemize}
			\item \textbf{来源 A (心中的推算)}:
			      你默数步数,假设每步 0.75 米。
			      \newline $\rightarrow$ \alert{困境}:步长其实不均匀,不知不觉就走歪了。
			      \vspace{0.3cm}
			\item \textbf{来源 B (外界的感知)}:
			      朋友在终点喊你,或者你偶尔伸手摸到了墙。
			      \newline $\rightarrow$ \alert{困境}:声音有回声,墙面凹凸不平,定位并不精准。
		\end{itemize}

		\column{0.4\textwidth}
		\begin{alertblock}{大脑的纠结}
			大脑此刻疯狂思考:
			\begin{itemize}
				\item \textbf{推算}说:“按步数,我已经到了!”
				      \newline \small{(盲目自信,误差随时间发散)}
				      \vspace{0.1cm}
				\item \textbf{感知}说:“听声音,好像还没到?”
				      \newline \small{(充满噪声,容易受干扰)}
			\end{itemize}
		\end{alertblock}
		\centering
		\vspace{0.2cm}
		\textbf{如何走出最直的线?}
	\end{columns}
\end{frame}

% --- 第二页:历史背景与提出者(阿波罗时刻) ---
\begin{frame}
	\frametitle{历史的转折}

	\begin{columns}
		\column{0.55\textwidth}
		\textbf{背景危机}:
		\begin{itemize}
			\item 阿波罗飞船必须以极高精度切入月球轨道。
			\item 机载计算机算力极低(仅2KB RAM),存不下传统方法所需的历史数据。
		\end{itemize}

		\vspace{0.3cm}
		\textbf{卡尔曼滤波的提出}:
		\begin{itemize}
			\item \textbf{Rudolf E. Kalman} 访问 NASA Ames 中心。
			\item \textbf{1960年发表论文}:
			      \newline \textit{\small "A New Approach to Linear Filtering and Prediction Problems"}
			\item \textbf{核心突破}:“只看现在,忘掉过去”,完美适配低算力环境。
		\end{itemize}

		\column{0.45\textwidth}
		% --- 图片区域 ---
		\begin{figure}
			\centering
			% 建议找一张卡尔曼的黑白肖像照
			\includegraphics[width=0.45\linewidth]{figures/kalman-portrait.png}
			% \includegraphics[width=0.45\linewidth]{example-image-a} % 占位符A,请替换
			\hfill
			% 建议找一张阿波罗登月舱或宇航员的照片
			% \includegraphics[width=0.45\linewidth]{apollo_moon.jpg}
			\includegraphics[width=0.45\linewidth]{figures/apollo.jpg} % 占位符B,请替换
			\caption{\scriptsize 左:R.E. Kalman | 右:阿波罗登月}
		\end{figure}

		\vspace{-0.2cm}

		\begin{block}{历史里程碑}
			卡尔曼滤波成为了阿波罗导航计算机的灵魂,不仅把人送上了月球,也把他们带了回来。
		\end{block}
	\end{columns}
\end{frame}

% --- 第三页:学科定义与影响(拔高) ---
\begin{frame}
	\frametitle{从控制到信息:卡尔曼滤波的本质}

	\begin{itemize}
		\item \textbf{它不是传统意义上的“滤波器”}:
		      \begin{itemize}
			      \item 它不只是滤除高频杂波,它是一个 \textbf{最优状态估计器}。
		      \end{itemize}

		\item \textbf{对信息工程的深远影响}:
		      \begin{enumerate}
			      \item \textbf{打破壁垒}:完美融合了控制理论中的状态空间模型与统计信号处理中的(贝叶斯估计)。
			      \item \textbf{定义标准}:确立了“预测 + 更新”的标准范式。
			      \item \textbf{无处不在}:
			            \begin{itemize}
				            \item \small{手机定位 (GPS + 加速度计融合)}
				            \item \small{大疆无人机悬停 (IMU + 视觉融合)}
				            \item \small{ChatGPT (Transformer 中的位置编码逻辑与序列预测思想亦有相通之处)}
			            \end{itemize}
		      \end{enumerate}
	\end{itemize}
\end{frame}